\documentclass[border=10pt,preview]{standalone}

% Language setting
% Replace `english' with e.g. `spanish' to change the document language
\usepackage[english]{babel}

% Set page size and margins
% Replace `letterpaper' with `a4paper' for UK/EU standard size
\usepackage[letterpaper,top=2cm,bottom=2cm,left=3cm,right=3cm,marginparwidth=1.75cm]{geometry}

% Useful packages
\usepackage{amsmath}
\usepackage{amssymb}
\usepackage{graphicx}
\usepackage{stmaryrd}
\usepackage{nicematrix}
\usepackage{ifthen}
\usepackage{minted}
\usepackage[colorlinks=true, allcolors=blue]{hyperref}

% DO NOT REMOVE THIS LINE
\newif\ifcorrection

\begin{document}

\section{Programmation dynamique}

\subsection{Triangle de Pascal}

L'objectif de cet exercice est de calculer le coefficient binomial $\forall n \in \mathbb{N}, \forall k \in \llbracket 0; n \rrbracket \quad \binom{n}{k}$.
\begin{enumerate}
    \item Donner les valeurs de $\forall n \in \mathbb{N}, \binom{n}{n}$ et $\binom{n}{0}$.
    \item Rappeler la formule du triangle de Pascal.
    \item Proposer un algorithme récursif permettant de calculer le coefficient binomial $\binom{n}{k}$.
    \item Déterminer la complexité de votre algorithme. Exposer un problème de l'algorithme récursif.
    \item Proposer un algorithme de programmation dynamique permettant de déterminer $\forall n \in \mathbb{N}, \forall k \in \llbracket 0; n \rrbracket \binom{n}{k}$
    \item Donner la complexité temporelle et spatiale de votre algorithme.
\end{enumerate}


% Example on how to show a part only in correction mode
\ifcorrection
\subsection*{Correction : Triangle de Pascal}
\begin{enumerate}
    \item {
\begin{minted}[samepage]{c}
int binom(int n, int k){
    if(n == 0) return 1;
    if(k == 0 || k == n) return 1;
    return binom(n-1, k) + binom(n-1, k+1);
}
\end{minted}
    }
    \item $\forall n \in \mathbb{N}, \binom{n}{0} = \binom{n}{n} = 1$
    \item $\forall n \in \mathbb{N}, \forall k \in \llbracket 0; n \rrbracket, \binom{n}{k} = \binom{n}{k} + \binom{n}{k-1}$
    \item {
    \item {
    On cherche à déterminer la complexité temporelle de notre algorithme, on se concentre sur le nombre d'additions faîtes pour atteindre l'étape $(n, k)$. On note cette quantité $c(n, k)$.

    D'après la formule de récurrence on a dans le pire cas :
    \begin{equation*} \label{}
    \begin{split}
        c(n, k) & = c(n-1, k) + c(n-1, k-1) + 1 \\
                & \leq 2 \times c(n-1, k) + 1
    \end{split}
    \end{equation*}

    Pour justifier l'inégalité on rappelle que l'on s'intéresse à la complexité pire cas et que dans le pire des cas le calcul s'arrête quand $n = 0$. On peut s'intéresser uniquement à $n$ et ainsi écrire $c_n = 2 * c_{n-1} + 1$.

    On rappelle que le $+1$ correspond aux opérations élémentaires réalisés à chaque étape, c'est donc un ordre de grandeur (on pourrait discuter longtemps du fait que l'on ne fait pas une unique opération élémentaire à chaque étape mais cela ne nous intérresse pas vraiment)

    On cherche alors l'expression générale d'une suite arithmético-géométrique : 
    
    \begin{equation*} \label{}
    \begin{split}
      c_n & = 2 * c_{n-1} + 1 \\
      - l & = 2*l + 1 \implies l = -1 \\
      c_n + 1 &= 2*c_{n-1} + 2 = 2*(c_{n-1} + 1) = 2^n (c_0 + 1)
    \end{split}
    \end{equation*}

    On obtient ainsi que la complexité temporelle est en $\mathcal{O}(c_n) = \mathcal{O}(2^n)$. Le problème de l'algorithme récursif c'est que de nombreuses valeurs sont calculés plusieurs fois.
    }
\begin{minted}[samepage]{c}
int binom(int n, int k){
    int memory[n][n];

    for(int i = 0; i < n; i++) memory[i][0] = memory[i][i] = 1

    for(int i = 1; i < n; i++){
        for(int j = 1; j < i-1; j++){
            memory[i][j] = memory[i-1][j] + memory[i-1][j-1];
        }
    }

    return memory[n][k]
}
\end{minted}
    }

    \item{
        On effectue un calcul par case de notre tableau \verb|memory|, le tableaux possède $n^2$ cases. La complexité temporelle et spatiale de notre algorithme est donc en $\mathcal{O}(n^2)$
    }
\end{enumerate}


\fi

\end{document}